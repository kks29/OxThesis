\chapter{\label{ch:3}Application of Spatial Transcriptomic Technologies in Breast Cancer}

\minitoc

\section{Introduction}


\section{Results}
%\label{app:imaging}


\subsection{Comparison of Emerging Spatial Transcriptomic Technologies}
Slide-seq
HDST
NanoString

spatial resolution vs cost vs availability

Biological systems are characterised by highly coordinated spatio-temporal gene expression patterns. Spatial patterns of gene expression are intricately linked with tissue identity and enable coordinated responses to micro-environmental stressors.

Spatial transcriptomics is an emerging domain within the life sciences which enables gene expression detection whilst retaining the spatial location of RNA transcripts. The available techniques are divided broadly into three approaches: 										\
i. Microdissected regions of interest														\
ii. Fluorescence in situ hybridisation (FISH) based methods. 								\
iii. Modifications of single cell sequencing techniques while allow Whole transcriptomic coverage. 																					\									

Microdissection can be performed using laser capture. Tissue sections are prepared onto uncharged membrane glass slides which facilitates tissue extraction of dissected regions. Tissue is stained with cresyl violet to aid visualisation during dissection. The laser microdissection unit is paired with a microscopy unit so that regions of interest can be identified and catapulted into capture tubes for RNA capture. This brute-force approach offers low sample-throughput and is not amenable for wide scale coverage of tissue. High resolution capture is particularly problematic since it is difficult to achieve reliable extraction of single cells. The downstream RNA extraction process for low input RNA also requires significant protocol optimisation in a tissue specific manner.

In contrast, in situ hydridisation approaches enable direct visualisation of RNA molecules within their native environment. Tissue is fixed after which mRNA is first reverse transcribed to cDNA. 
Fluorescently labelled padlock probes complementary to the target of interest can then bind to the generated cDNA. Target amplification, required for downstream detection, is achieved by a process called rolling-circle amplification (RCA). RCA generates multiple copies of the padlock probe with inter-probe gaps filled by sequencing-by-ligation. The end product is imaged and decoded in a manner analogous to Illumina sequencing-by-synthesis. CARTANA and ReadCoor, companies acquired by 10X Genomics, offer commercial access to in situ hybridisation platforms.

Lastly, spatial transcriptomics can be performed by in situ transcript capture followed by ex situ sequencing. There are a range of available options which generate a spatially profiled glass slide through different approaches. The Visium option entails a glass slide onto which a barcoded RT primer is printed. The barcoded primer encodes the x and y coordinate position of the array. Tissue is fixed, stained, imaged and permeabilised onto the array. Permeabilised mRNA hybridises to the barcoded primers and in situ RT is performed. The barcoded reads are then mapped back to the original tissue location following completion of library preparation and next generation sequencing. Visium offers limited spatial resolution: spatially profiled regions are 55 $\mu$m diameter circular spots.

Slide-seq and High-Definition Spatial Transcriptomics (HDST) are conceptually similar to Visium. Polystyrene beads, each of 10 $\mu$m diameter, are attached in monolayer onto a glass coverslip in place of printing barcoded RT primers onto a glass slide. Each bead has a randomly attached unique spatial barcode. The bead location is not known a priori. Therefore, bead position is decoded in situ using sequencing-by-ligation. Sample preparation, permabilisation and library generation are similar to Visium. Slide-seq spatial resolution, governed by bead diameter, is higher than Visium and is akin to single cell resolution. HDST is extremely similar to Slide-seq except 2 $\mu$m diameter beads are used.

Slide-seq and HDST are only suitable for use on fresh frozen tissue. An analogous approach suitable for FFPE is provided by the Nanostring GeoMx Digitial Spatial Profiler.

Visium
Slide-seq/HDST
GeoMx

Set of factors: availabile, cost, tissue suitability.

\subsection{Application of Slide-seq in Breast Cancer}
\subsubsection{Development of Slide-seq Pipeline in Breast Cancer Biopsy Samples}

After careful consideration and review of all spatial transcriptomic technologies, I elected to proceed with the use of Slide-seq on treatment naive fresh frozen breast cancer biopsies. Slide-seq is a novel technology which is not routinely available within Europe. The technology cannot be purchased from a commercial provider. At present, it is only available within the Macosko or Regev lab at the Broad Institute of MIT and Harvard, Boston, USA.


Therefore, I independently established a collaboration directly with the Macosko lab in order to proceed with this research program for my DPhil. I also established a bespoke training program within the Macosko lab for which I travelled to Boston, USA in November 2019. The training objectives were:

a. To develop expertise in the production of spatially barcoded beads and the effective uniform placement of beads onto glass slides. \\
b. To understand the process by which each bead is spatially profiled using in situ sequencing technology. \\
c. To develop best practice techniques in sample selection, tissue placement onto spatial slides and the downstream processing for library generation.

Upon completion of the training program in the US, I returned back to Oxford with a collection of pilot spatial slides to set up the novel experimental pipeline with the associated specialised reagents. The establishment of the in situ sequencing platform for the spatial slides required several years of optimisation by the Macosko lab. It would require \textasciitilde 1-2 years to develop an equivalent optimised platform in Oxford. My priority for the DPhil was to explore the application of spatial transcriptomic technologies in clinical breast cancer samples. Therefore, I aimed to establish a pipeline in which the fully profiled spatial slides would be prepared by the Macosko lab and I would process the spatial slides on clinical breast cancer samples in Oxford.

Each spatial slide, referred to as a 'puck', is produced from a 0.17 mm thick glass coverslip which is fragile. I developed expertise in the correct and gentle handling of the coverslips to avoid the creation of cracks during the protocol. Cracks created during the tissue placement stage, in particular, would introduce spatial artefacts. Consequently, the correct selection of tweezer for handling of the pucks was essential. An iterative selection process using multiple pilot slides allowed the identification of a fine stainless steel tweezer (\#T4537-1EA Sigma) as optimal for puck handling.

Slide-seq is suitable only for fresh frozen tissue. The sectioning of fresh frozen tissue is performed within a cryostat. Standard practice entails embedding tissue within OCT to provide a stable base from which sectioning is performed. However, OCT interferes with RNA hybridisation onto the puck. The placement of tissue within a cryostat was accordingly modified as part of the Slide-seq protocol. A small volume of OCT is placed to allow for tissue stabilisation. However, the upper layer of the tissue specimen is kept free of OCT. 

Breast cancer clinical biopsies are \textasciitilde 4 mm diameter and of variable shape and adipose composition. The physical characteristics of the biopsies presented significant additional challenges for accurate tissue placement onto the puck. Tissue sectioning and placement was performed with Dr. Esther Bridges, a very senior and experienced scientist. The skill and technical acumen of Dr. Bridges was an essential component required for the successful establishment of the Slide-seq pipeline.


\begin{figure}[!htb]
	\minipage{0.32\textwidth}
		\left
		\includegraphics[width=0.7\textwidth, angle=270]{figures/slideseq_cryostat_1.jpg} \hfill
		\caption[Biopsy orientation in cryostat]{Biopsy \\ orientation in cryostat}
		\label{fig:slideseq_cryostat_1}
	\endminipage\hfill
	\minipage{0.32\textwidth}
		\centering
		\includegraphics[width=0.7\textwidth, angle=270]{figures/slideseq_cryostat_2.jpg} \hfill
		\caption[Sample 1]{Sample 1}
		\label{fig:slideseq_cryostat_2}
	\endminipage\hfill
	\minipage{0.32\textwidth}
		\centering
		\includegraphics[width=0.7\textwidth, angle=270]{figures/slideseq_cryostat_3.jpg} \hfill
		\caption[Sample 2]{Sample 2}
		\label{fig:slideseq_cryostat_3}
	\endminipage
\end{figure}




\subsubsection{Optimisation of Slide-seq Protocol for Application in Breast Cancer}
The optimised Slide-seq protocol has been published for investigation of murine brain tissue. Breast cancer tissue exhibits distinct histological differences compared to normal brain tissue: marked cellular heterogeneity, variable fibroblast composition, incoherence in spatial architecture. I therefore optimised the Slide-seq protocol for use in breast cancer, with a specific focus on optimising RNA hydribidisation onto the spatial beads. RNA hybridisation involves conflicting factors: sufficient duration to allow for adequate RNA capture whilst maintaining adequate spatial resolution without excessive lateral diffusion of RNA.

In murine brain tissue, RNA hydridisation is performed for 15 minutes. In breast tissue, I explored a range of durations for RNA hybridisation: 15, 30, 45, 60, 75, 90 and 105 minutes.

Clinical tumour tissue is valuable. The use of clinical tissue must fulfil additional ethical and practical requirements in using patient donated material. Therefore, for protocol optimisation, I used a xenograft sample acquired from Prof. Harris' extensive tissue biobank. The xenograft sample selected was generated from MDA-MB-231 breast cancer cell line. a triple receptor breast cancer cell line, treated with PBS in BALB/c nude mice (experiment ID JLI-164, sample 6).

Each xenograft sample consists of a FFPE and fresh frozen matched pair. Sample selection was based on high carbonic anhydrase 9 (CA9) expression in FFPE tissue and the downstream Slide-seq protocol was conducted on the matched fresh frozen tissue. FFPE preservation allows for better maintenance of spatial architecture which facilitated sample selection. An optimised protocol for CA9 immunohistochemistry on FFPE tissue was also readily available.

The quality of RNA hydridisation onto the spatially-profiled beads was assessed using the Agilent high sensitivity DNA kit prior to library preparation. RNA hydridisation increased up to peak quality observed with 60 minutes hydridisation, after which RNA capture quality demonstrated variable decline [IMAGES OF BIOANALYZER RESULTS]. Therefore, all subsequent Slide-seq experiments in breast cancer tissue used the modified 60 minutes RNA hydridisation condition.

The slide-seq optimisation experiment was completed in March 2020. All work stopped in March 2020 owing to the UK national lockdown due to the COVID-19 pandemic. The laboratory areas started to re-open in June 2020. Between June 2020 - December 2020, I explored multiple diverse avenues to enable the Slide-seq work to restart. The pre-pandemic experiments required working on multiple sites, using critical equipment housed in different institutions and coordinating with team members from within and outside my group. In January 2021, it was possible to attempt one run of the Slide-seq experiment. The experimental timing coincided with the beginning of the third UK national lockdown. 

Four fresh frozen breast cancer biopsy specimens acquired from three patients were utilised for spatial transcriptomic profiling using the breast cancer optimised protocol. Twenty breast cancer specimens were evaluated for general histological quality by hematoxylin and eosin  staining from which the most suitable samples were selected. The H{\&}E evaluation of the samples are shown in figure XX [H{\&} OF SLIDESEQ BIOPSIES].

Good quality sequencing libraries were generated from all four biopsies [BIOANALZYER IMAGES POST NEXTERA LIBRARY GENERATION]. Sequencing has been completed. Sequencing data was transferred to the Macosko lab, Broad Institute in March 2021 for the generation of spatially-profiled gene expression matrices suitable for downstream analysis and evaluation. The return of data is being awaited and will warrant analysis once available.

In April 2021, I had arranged access to an additional collection of Slide-seq spatial slides to enable spatial transcriptomic evaluation of a larger collection of samples. 
Dr. Esther Bridges was a key research scientist involved in my project who conducted tissue sectioning onto the spatial slides. Dr. Bridges has transferred to a new post and no alternative options were available. It was therefore not possible to proceed further with the additional set of spatial slides.


\subsubsection{Modelling of Bead Packing Density in Slide-seq}
The technology platforms underpinning the emerging field of spatial transcriptomics are continuing to evolve. Most options to date are produced or performed within academic settings with significant costs per sample. The maximisation of output and the availability of quality control metrics for spatial slide production will become increasingly important as spatial transcriptomic techniques become more widely adopted by the research community.

Slide-seq pucks consist of 3 mm diameter circles consisting of packed 10 $\mu$m custom circular polysytrene beads deposited within the central circular region. Thue's theorem states that 'no packing of non-overlapping discs of equal size in the plane has density higher than that of the hexagonal packing'. The optimal packing density is $\frac{\pi \sqrt{3}}{6} = 0.91$. Thue's theorem is an extension of Kepler's conjecture in the 2D setting. An unresolved debate remains within mathematical community on the demonstration of a formal proof of Kepler's conjecture.

I calculated the total number of beads per puck using the particle counting package in ImageJ. Using the assumption that each polysytrene bead is a 10 $\mu$m circle and the puck is a 3 mm circle, I calculated the packing efficiency for each bead. The per puck and per batch summary statistics are detailed in tables [XX] [TABLE OF SUMMARY STATS].


Future follow-up work would include evaluation of the packing efficiency threshold as a quality control metric applicable in industrial production of pucks. The number of beads per puck could also be integrated as a normalisation factor for the downstream analysis pipeline.

\subsubsection{Analysis of Slide-seq Profiling in Breast Cancer}
Data required for the completion of this section is pending owing to the exceptional delays and disruption caused by the pandemic. 

\subsubsection{Application and Comparison of Hypoxia Signatures in Slide-seq Findings}
Data required for the completion of this section is pending owing to the exceptional delays and disruption caused by the pandemic. 

\subsubsection{Identification of Tumour and Immune-Specific Hypoxia Markers}
Data required for the completion of this section is pending owing to the exceptional delays and disruption caused by the pandemic. 

\subsection{Application of Single Cell Laser Capture Microdissection in Breast Cancer}
\subsubsection{Tumour and Immune Extraction in Breast Cancer at Single Cell Resolution}
Data required for the completion of this section is pending owing to the exceptional delays and disruption caused by the pandemic. 

\subsubsection{Analysis of Single Cell RNA Sequencing}
Data required for the completion of this section is pending owing to the exceptional delays and disruption caused by the pandemic. 

\subsubsection{Comparison of Slide-seq with Single Cell Laser Capture Microdissection in Serial Sections}
Data required for the completion of this section is pending owing to the exceptional delays and disruption caused by the pandemic. 

\subsection{Application of In Situ Sequencing in Breast Cancer}
\subsubsection{Application of In Situ Sequencing by CARTANA on Breast Cancer}
In order to complement the ex situ sequencing approach offered by the Slide-seq technology, I explored in situ sequencing technologies provided by ReadCoor and CARTANA during the course of the DPhil. This was a major undertaking since in situ sequencing is highly specialised and requires bespoke equipment. The ReadCoor technology is not yet available in Europe.

In order to address these challenges and gain access to the platforms, I independently established a collaboration with ReadCoor, a start-up technology company based in Boston, USA. I negotiated a pilot study in which two breast cancer samples would be processed free-of-charge by the ReadCoor team in Boston and established the legal and material transfer agreement between ReadCoor and Oxford University. The pilot samples were shipped to Boston in early December 2019. The study was due to start in February 2020 by which stage ReadCoor had closed its operations due to the pandemic and the study was on hold for a significant proportion of 2020. 

In September 2020, ReadCoor was acquired by 10X Genomics which required a new legal agreement between 10X Genomics and Oxford University. After extensive renegotiation with 10X Genomics, the pilot study was transferred to CARTANA, a different subsidiary of 10X Genomics offering a similar spatial transcriptomic platform. The pilot samples were shipped to CARTANA, Sweden in December 2020. 

In addition to a tumour-immune panel, I also negotiated a personalised hypoxia-specific 48-gene panel as part of the study review process with CARTANA. The hypoxia panel was optimised free-of-charge by the CARTANA technical team in Sweden for breast cancer FFPE tissue. 

I used an evidence based approach for target selection in the hypoxia specific panel using my in-house unpublished and published \cite{2017_nat_comm_chung} breast cancer single cell sequencing data. Genetic targets which fulfilled the following two features were included in the hypoxia panel: \
i. Targets exhibiting a predominant non-zero distribution pattern.
A major challenge in single cell datasets is the commonly observed zero-inflated distribution expression patterns.
ii. Targets exhibiting a non-uniform distribution profile in order to allow for investigation of spatial heterogeneity.

Figures X and X demonstrate expression profiles of selected and deselected genetic targets which are associated with the transcriptional response to hypoxia.

Target selected: PGK1
Target deselected: PFKFB4

Samples were selected for the pilot study based on carbonic anhydrase 9 (CA9) expression. All samples were collected from the Oxford Radcliffe Biobank collection, screened by hormone receptor and HER2 status and only triple receptor negative breast cancer samples were selected. Samples BBP-1467 and BBP-1557 [FIG OF CA9 IHC] exhibited well-circumscribed regions of hypoxia as defined by CA9 expression and were therefore selected for the pilot study.

The experimental component of this project was completed by the end of March 2021.


\subsubsection{Analysis of In Situ Sequencing Data in Breast Cancer}
The analysis of the CARTANA data is at a very early stage and it is not available for incorporation into the thesis owing to the exceptional delays and disruption caused by the pandemic. 

\subsection{Comparison of Expression of Hypoxia Signatures upon Spatial Transcriptomic Profiling in Breast Cancer}
Data required for the completion of this section is pending owing to the exceptional delays and disruption caused by the pandemic. 

\section{Discussion}
I made steps to select most appropriate emerging spatial transcriptomic technology and apply these methods to clinical breast cancer samples. 
Further insights can be gained once the data is available or preliminary analysis has been completed on the available data.


